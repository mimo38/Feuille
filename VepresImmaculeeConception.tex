% Afficher des recommandations concernant la syntaxe.
\RequirePackage[orthodox,l2tabu]{nag}
\RequirePackage{luatex85}
% Paramètres du document.
\documentclass[%
a5paper%                       Taille de page.
,11pt%                         Taille de police.
,DIV=16%                       Plus grand => des marges plus petites.
,titlepage=off%                 Faut-il une page de titre ?
%,headings=optiontoheadandtoc%  Effet des paramètres optionnels de section.
%,headings=small%
,parskip=false%
,openany%
]{scrbook}
%\renewcommand*\partheademptypage{\thispagestyle{empty}}
\newcounter{facteur}\setcounter{facteur}{17}%%%%%%%%%%%%%% Paramètre pour la taille globale des partitions. par défaut : 17
%\usepackage{geometry}
\usepackage{gredoc,mudoc,lyluatex}
\usepackage{pdfpages,transparent,array,ltablex}
\usepackage{framed}

%%%%%%%%%%%%%%%%%%%%%%% Paramètres variables %%%%%%%%%%%%%%%%%%%%%%%%%%%%%%%%%%%%%%%%%%%%%%%%%%
%%% Taille des partitions grégoriennes.                                                      %%
%\grechangedim{overhepisemalowshift}{.7mm}{scalable} %%Distance to place a a horizontal episema over a note in a low position in the space.Default: 0.02287 cm
%\grechangedim{hepisemamiddleshift}{1.4mm}{scalable} %%Distance to place a horizontal episema in the middle of a space. Default: 0.07206 cm
%\grechangedim{overhepisemahighshift}{2.1mm}{scalable} %% Distance to place a horizontal episema over a note in a high position in the space. Default: 0.10066 cm
%\grechangedim{vepisemahighshift}{2.1mm}{scalable} %% Distance to place a vertical episema in a high position in the space. Default: 0.06634 cm
%\grechangestafflinethickness{50} %%% epaisseur des lignes The default value is same as staff size.
\grechangestaffsize{\value{facteur}}%%%%%
%%%%%%%%%%%%%%%%%%%%%%%%%%%%%%%%%%%%%%%%%%%%%%%%%%%%%%%%%%%%%%%%%%%%%%%%%%%%%%%%%%%%%%%%%%%%%%%
% Par souci de clarté, la définition des commandes est reportée dans un document annexe.

%\addtolength{\voffset}{2mm}\addtolength{\headsep}{-2mm}
%\setlength{\extrarowheight}{2mm}
\addto\captionsfrench{%
  \renewcommand{\indexname}{Index des chants}%
}

\pdfcompresslevel=9

\def\arraystretch{1.2}

%%%%%%% Commandes ajoutées assez explicites %%%%%%%%%%%%%%%%
\newcommand{\reponsegras}[2]{
    \versio{\textbf{#1}}{{#2}}
}
\newcommand{\imagecentre}[2]{
\begin{center} \includegraphics[height=#1]{img/#2} \end{center}}

%%%%%%%%%%% Pour la page de titre %%%%%%%%%%%%%
\title{Vêpres de l'Immaculée Conception}
\author{}
\date{8 décembre}

%%%%%%%%%%%%%%%%%%%%%%%%%%%%%%%%%%%%%%%%%%%%%%%%%%%%%%%%%%%%%%%%%%%%%%%%%%%%%%%%
%%%%%%%%%%%%%%%%%%%%%%%%%%%%%%%%%%%%%%%%%%%%%%%%%%%%%%%%%%%%%%%%%%%%%%%%%%%%%%%%
%%%%%%%%%%%%%%%%%%%%%%%%%%%%%%%%%%%%%%%%%%%%%%%%%%%%%%%%%%%%%%%%%%%%%%%%%%%%%%%%
\begin{document}
                \newfontfamily\malettrine[Scale=0.8]{l800}
                \renewcommand{\LettrineFontHook}{\malettrine\color{black}}

                \newcommand{\ligne}[2]{
                \begin{center}
                \greseparator{#1}{#2}
                \end{center}
                }%
\maketitle
%\newpage
%\vspace*{\stretch{9}}
%\thispagestyle{empty}
%\vspace*{\stretch{1}}
%\newpage

\cantus{Verset}{DeusInAdiutorium_solemnis}{}{}
\medskip

\vulgo{Vous êtes toute belle, ô Marie, et la tache originelle n’est pas en vous.}
\cantus{Antienne}{TotaPulchraEs}{1.Ant}{1.g.2}
\rubrica{\textit{Ps. 109}}
\psalmus[numerus=1]{109-1g2}
\gloria[tonus=1g2]
\cantus{Antienne}{TotaPulchraEs}{1.Ant}{1.g.2}

\medskip

\vulgo{Votre vêtement est blanc comme la neige, et votre visage comme le soleil.}
\cantus{Antienne}{VestimentumTuum}{2.Ant}{8.G}
\rubrica{\textit{Ps. 112}}
\psalmus[numerus=1]{112-8G}
\gloria[tonus=8G]
\cantus{Antienne}{VestimentumTuum}{2.Ant}{8.G}

\medskip

\vulgo{Vous êtes la gloire de Jérusalem, vous êtes la joie d’Israël, vous êtes l’honneur de notre peuple.}
\cantus{Antienne}{TuGloria}{3.Ant}{8.c}
\rubrica{\textit{Ps. 121}}
\psalmus[numerus=1]{121-8c}
\gloria[tonus=8c]
\cantus{Antienne}{TuGloria}{3.Ant}{8.c}

\medskip
\vulgo{Vous êtes bénie, ô Vierge Marie, par le Seigneur, le Dieu très haut, plus que toutes les femmes, sur la terre.}
\cantus{Antienne}{BenedictaEsTu}{4.Ant}{7.a}
\rubrica{\textit{Ps. 126}}
\psalmus[numerus=1]{126-7a}
\gloria[tonus=7a]
\cantus{Antienne}{BenedictaEsTu}{4.Ant}{7.a}

\medskip
\vulgo{Entraînez-nous, Vierge immaculée, nous courrons à votre suite, à l’odeur de vos parfums.}
\cantus{Antienne}{TraheNos}{5.Ant}{3.a2}
\rubrica{\textit{Ps. 147}}
\psalmus[numerus=1]{147-3a2}
\gloria[tonus=3a2]
\cantus{Antienne}{TraheNos}{5.Ant}{3.a2}

\subsection{Capitule}
\versio{Dóminus possédit me in inítio viárum suárum, ántequam quidquam fáceret a princípio. Ab ætérno ordináta sum, et ex antíquis ántequam terra fíeret. Nondum erant abýssi, et ego iam concépta eram.}
{Le Seigneur m’a possédée au commencement de ses voies, avant tout ce qu’il a fait dès le principe. J’ai été établie dès l’éternité, et dès les temps anciens, avant que la terre fût créée. Les abîmes n’étaient pas encore, que j’étais déjà conçue.}
\reponsegras{%
℟. Deo grátias.%
}{%
℟. Rendons grâces à Dieu.%
}

%\cantus{Hymne}{AveMarisStella}{Hymn.}{1}
\subsection{Hymne}
\input{gabc/Hymne/AveMarisStellaComplet.tex}
\medskip
\versio{℣. Immaculáta Conceptio est hódie sanctæ Maríæ Vírginis.}{℣. C’est aujourd’hui l’immaculée Conception de la sainte Vierge Marie.}
\reponsegras{℟. Quæ serpéntis caput virgíneo pede contrívit.}{℟. Qui a écrasé la tête du serpent de son pied virginal.}

\ligne{2}{10}
\rubrica{aux premières vêpres :}
\cantus{Antienne}{BeatamMeDicent}{Ant.}{8G}
%\canticum[tonus=8G,primus=1,numerus=1]{Magnificat}
%\gloria[tonus=8G]
\ligne{2}{10}

\rubrica{aux deuxièmes vêpres :}
\vulgo{Aujourd’hui est sorti un rejeton de la tige de Jessé ; aujourd’hui Marie a été conçue sans aucune tache de péché ; aujourd’hui a été écrasée, par elle, la tête de l’antique serpent, alléluia.}
\cantus{Antienne}{HodieEgressaEst}{Ant.}{1f}
\ligne{2}{10}
\canticum[tonus=1f,primus=1,numerus=1]{Magnificat}
\gloria[tonus=1f]

\subsection{Oraison}

\oratio{%
Deus, qui per immaculátam Vírginis Conceptiónem dignum Fílio tuo habitáculum præparásti: quǽsumus; ut qui ex morte eiúsdem Fílii tui prævísa, eam ab omni labe præservásti, nos quoque mundos eius intercessióne ad te perveníre concédas.
Per eúndem Dóminum}{%
Ô Dieu, qui, par l’immaculée Conception de la Vierge, avez préparé à votre Fils une demeure digne de lui, et qui, en prévision de la mort de ce même Fils, avez préservé celle-ci de toute souillure, accordez-nous, nous vous le demandons, d’arriver jusqu’à vous, purs nous aussi, par son intercession.
Par le même Jésus Christ%
}

\subsection*{Conclusion}

\versio{%
℣. Dóminus vobíscum.%
}{%
℣. Le Seigneur soit avec vous.%
}

\reponsegras{%
℟. Et cum spíritu tuo.%
}{%
℟. Et avec votre esprit.%
}

\medskip
\rubrica{aux premières vêpres :}
\cantus{Verset}{BenedicamusDominoIVesp}{}{}
\rubrica{aux deuxièmes vêpres :}

\cantus{Verset}{BenedicamusDomino-IIVepres-IClasse-AdLib}{}{}
%\versio{Fidélium ánimæ per misericórdiam Dei requiéscant in pace.}{Que par la miséricorde de Dieu, les âmes des fidèles trépassés reposent en paix.}
%\reponsegras{℟. Amen.}{℟. Ainsi soit-il.}
\end{document}
