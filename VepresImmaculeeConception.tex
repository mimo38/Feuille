% Afficher des recommendations concernant la syntaxe.
\RequirePackage[orthodox,l2tabu]{nag}
\RequirePackage{luatex85}
% Paramètres du document.
\documentclass[%
a5paper%                       Taille de page.
,11pt%                         Taille de police.
,DIV=16%                       Plus grand => des marges plus petites.
,titlepage=off%                 Faut-il une page de titre ?
%,headings=optiontoheadandtoc%  Effet des paramètres optionnels de section.
%,headings=small%
,parskip=false%
,openany%
]{scrbook}
%\renewcommand*\partheademptypage{\thispagestyle{empty}}
\newcounter{facteur}\setcounter{facteur}{17}%%%%%%%%%%%%%% Paramètre pour la taille globale des partitions. par défaut : 17
%\usepackage{geometry}
\usepackage{gredoc,mudoc,lyluatex}
\usepackage{pdfpages,transparent,array,ltablex}
\usepackage{framed}

%%%%%%%%%%%%%%%%%%%%%%% Paramètres variables %%%%%%%%%%%%%%%%%%%%%%%%%%%%%%%%%%%%%%%%%%%%%%%%%%
%%% Taille des partitions grégoriennes.                                                      %%
%\grechangedim{overhepisemalowshift}{.7mm}{scalable} %%Distance to place a a horizontal episema over a note in a low position in the space.Default: 0.02287 cm
%\grechangedim{hepisemamiddleshift}{1.4mm}{scalable} %%Distance to place a horizontal episema in the middle of a space. Default: 0.07206 cm
%\grechangedim{overhepisemahighshift}{2.1mm}{scalable} %% Distance to place a horizontal episema over a note in a high position in the space. Default: 0.10066 cm
%\grechangedim{vepisemahighshift}{2.1mm}{scalable} %% Distance to place a vertical episema in a high position in the space. Default: 0.06634 cm
%\grechangestafflinethickness{50} %%% epaisseur des lignes The default value is same as staff size.
\grechangestaffsize{\value{facteur}}%%%%%
%%%%%%%%%%%%%%%%%%%%%%%%%%%%%%%%%%%%%%%%%%%%%%%%%%%%%%%%%%%%%%%%%%%%%%%%%%%%%%%%%%%%%%%%%%%%%%%
% Par souci de clarté, la définition des commandes est reportée dans un document annexe.

%\addtolength{\voffset}{2mm}\addtolength{\headsep}{-2mm}
%\setlength{\extrarowheight}{2mm}
\addto\captionsfrench{%
  \renewcommand{\indexname}{Index des chants}%
}

\pdfcompresslevel=9

\def\arraystretch{1.2}

%%%%%%% Commandes ajoutées assez explicites %%%%%%%%%%%%%%%%
\newcommand{\reponsegras}[2]{
    \versio{\textbf{#1}}{{#2}}
}
\newcommand{\imagecentre}[2]{
\begin{center} \includegraphics[height=#1]{img/#2} \end{center}}

%%%%%%%%%%% Pour la page de titre %%%%%%%%%%%%%
\title{Vêpres de l'Immaculée Conception}
\author{}
\date{8 décembre}

%%%%%%%%%%%%%%%%%%%%%%%%%%%%%%%%%%%%%%%%%%%%%%%%%%%%%%%%%%%%%%%%%%%%%%%%%%%%%%%%
%%%%%%%%%%%%%%%%%%%%%%%%%%%%%%%%%%%%%%%%%%%%%%%%%%%%%%%%%%%%%%%%%%%%%%%%%%%%%%%%
%%%%%%%%%%%%%%%%%%%%%%%%%%%%%%%%%%%%%%%%%%%%%%%%%%%%%%%%%%%%%%%%%%%%%%%%%%%%%%%%
\begin{document}
    \newfontfamily\malettrine[Scale=0.8]{l800}
    \renewcommand{\LettrineFontHook}{\malettrine\color{black}}

    \newcommand{\ligne}[2]{
    \begin{center}
    \greseparator{#1}{#2}
    \end{center}
}%
\maketitle
%\newpage
%\vspace*{\stretch{9}}
%\thispagestyle{empty}
%\vspace*{\stretch{1}}
%\newpage
\section {Vêpres de l'Immaculée Conception}

\cantus{Verset}{DeusInAdiutorium_solemnis}{}{℣.}
\medskip

\cantus{Antienne}{TotaPulchraEs}{1.ANT.}{1.g.2}
\psalmus[numerus=1]{109-1g2}
\gloria[tonus=1g2]
\cantus{Antienne}{TotaPulchraEs}{1.ANT.}{1.g.2}

\medskip

\cantus{Antienne}{VestimentumTuum}{2.ANT.}{8.G}
\psalmus[numerus=1]{112-8G}
\gloria[tonus=8G]
\cantus{Antienne}{VestimentumTuum}{2.ANT.}{8.G}

\medskip

\cantus{Antienne}{TuGloria}{3.ANT.}{8.c}
\psalmus[numerus=1]{121-8c}
\gloria[tonus=8c]
\cantus{Antienne}{TuGloria}{3.ANT.}{8.c}

\medskip

\cantus{Antienne}{BenedictaEsTu}{4.ANT.}{7.a}
\psalmus[numerus=1]{126-7a}
\gloria[tonus=7a]
\cantus{Antienne}{BenedictaEsTu}{4.ANT.}{7.a}

\medskip

\cantus{Antienne}{TraheNos}{5.ANT.}{3.a2}
\psalmus[numerus=1]{147-3a2}
\gloria[tonus=3a2]
\cantus{Antienne}{TraheNos}{5.ANT.}{3.a2}

\newpage
\subsection{Ave Regina Cælorum}

\vulgo{Witaj, niebios Królowo,
Witaj, Pani aniołów,
Witaj, Różdżko, witaj Bramo,
Jasność zrodziłaś światu.

Ciesz się, Panno chwalebna,
Ponad wszystkie piękniejsza,
Witaj, o Najśliczniejsza,
Proś Chrystusa za nami.}


\cantus{Antienne}{AveReginaCaelorum-simple}{ANT.}{6.}
\ligne{2}{10}
\newpage

\subsection{Regina Cæli}

\vulgo{Królowo nieba, wesel się, alleluja,
Bo Ten, któregoś nosiła, alleluja,
Zmartwychwstał jak powiedział, alleluja,
Módl się za nami do Boga, alleluja.
}

\cantus{Antienne}{ReginaCaeli-simple}{ANT.}{6.}
\ligne{2}{10}
\newpage

\subsection{Salve Regina}

\vulgo{Witaj Królowo, Matko Miłosierdzia,
życie, słodyczy i nadziejo nasza, witaj!
Do Ciebie wołamy wygnańcy, synowie Ewy;
do Ciebie wzdychamy jęcząc i płacząc na tym łez padole.

Przeto, Orędowniczko nasza,
one miłosierne oczy Twoje na nas zwróć,
a Jezusa, błogosławiony owoc żywota Twojego,
po tym wygnaniu nam okaż.
O łaskawa, o litościwa, o słodka Panno Maryjo!
}

\cantus{Antienne}{SalveRegina-simple}{ANT.}{5.}
\ligne{2}{10}

\end{document}
